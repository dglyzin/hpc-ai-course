\documentclass[aspectratio=169]{beamer}
\usepackage[utf8]{inputenc}
\usepackage[russian]{babel}
\usepackage{tikz}
\usepackage{graphicx}
\usepackage{fontspec}
\usepackage{adjustbox}
\usepackage[inkscapeversion=1.2]{svg} 
\usetikzlibrary{shadings,shapes, arrows, positioning, fit}

\usepackage{pgfplots}
\pgfplotsset{compat=1.18}

% Define colors for clarity
\definecolor{docker}{RGB}{225,108,54}
\definecolor{slurm}{RGB}{63,81,181}
\definecolor{jupyter}{RGB}{239,108,0}
\definecolor{gpu}{RGB}{76,175,80}



\beamertemplatenavigationsymbolsempty

% Set the background color
% #000A1E
\definecolor{sitebackground1}{RGB}{0,10,30}
% #040714
\definecolor{sitebackground2}{RGB}{4,7,20}
% #66c6ea
\definecolor{logocolor}{RGB}{102,198,234}



\definecolor{darkblue}{RGB}{25,28,53}
\definecolor{baseblue}{RGB}{19,80,158}
\definecolor{additionalblue}{RGB}{47,127,193}

\definecolor{baselightblue}{RGB}{64,176,228}
\definecolor{additionallightblue}{RGB}{102,197,238}

\definecolor{baselightblue}{RGB}{64,176,228}

\setbeamercolor{background canvas}{bg=baseblue}

\setbeamercolor{normal text}{fg=white}
\setbeamercolor{title}{fg=white}
\setbeamercolor{frametitle}{fg=white}
\setbeamercolor{itemize item}{fg=logocolor}
\setbeamercolor{block title}{fg=logocolor}


% Load custom fonts
\newfontfamily\headerfont{Unageo}[
Path = ../fonts/,
Extension = .ttf,
UprightFont = *-Regular,
BoldFont = *-Bold,
ItalicFont = *-Regular-Italic,
]

\setsansfont{Montserrat}[
Path = ../fonts/,
Extension = .ttf,
UprightFont = *-Regular,
BoldFont = *-Bold,
ItalicFont = *-Italic,
]

% Set title and header font to Unageo
\setbeamerfont{title}{family=\headerfont, series=\bfseries}
\setbeamerfont{frametitle}{family=\headerfont, series=\bfseries}

% Command to easily adjust background images
\newcommand{\backgroundimages}[4]{%
	\begin{tikzpicture}[remember picture, overlay]
		% Left background image
		\node[anchor=south west, inner sep=0] at ([xshift=#1-0.3cm]current page.south west) 
		{\includesvg[width=0.3\paperwidth, height=0.3\paperheight]{#2}};
		
		% Right background image
		\node[anchor=north east, inner sep=0] at ([xshift=#1+0.3cm]current page.north east) 
		{\includesvg[width=0.3\paperwidth, height=0.3\paperheight]{#3}};
		
		% Center background image
		\node[anchor=south, inner sep=0] at ([xshift=#1-4cm]current page.south) 
		{\includesvg{#4}};
	\end{tikzpicture}%
}

\newcommand{\backgroundgradient}{%
	\begin{tikzpicture}[remember picture, overlay]		
		\node[anchor=south, inner sep=0] at ([xshift=-4cm]current page.south) 
		{\includesvg{Ellipse_4.svg}};
	\end{tikzpicture}%
}

\setbeamertemplate{background canvas}{
	\begin{tikzpicture}[remember picture, overlay]
		\shade[left color=baseblue, right color=darkblue] 
		(current page.south west) rectangle (current page.north east);
		
		\node[anchor=south west, inner sep=0] at ([xshift=-0.3cm]current page.south west) 
		{\includesvg[width=0.3\paperwidth, height=0.3\paperheight]{../img/Group_4.svg}};
		
		% Right background image
		\node[anchor=north east, inner sep=0] at ([xshift=0.3cm]current page.north east) 
		{\includesvg[width=0.3\paperwidth, height=0.3\paperheight]{../img/Group_5.svg}};
		
	\end{tikzpicture}
}

% Title page
\title{МАШИННОЕ ОБУЧЕНИЕ}
\subtitle{С СОВРЕМЕННЫМИ ПРИЛОЖЕНИЯМИ}
\author{Дмитрий Глызин}
\institute{Акселкомп / ЯрГУ}
%\date{\today}
\date{Весна 2026}

\begin{document}
	
	% First slide with title
	{
		%\setbeamertemplate{background}{
			%	\backgroundimages{0cm}{Group_4.svg}{Group_5.svg}{Ellipse_4.svg}
			%}
		\begin{frame}
			\titlepage
		\end{frame}
	}
	
	% Section slide
	\section{Introduction}
	{
	
		\begin{frame}
			\frametitle{ПЛАН КУРСА}
			
			Раздел 1. Практикум по применению больших языковых моделей
			
			Раздел 2. Нейросети
			
			Раздел 3. Классические алгоритмы
			
			
		\end{frame}
%%%%%%%


	\begin{frame}{Control Flow: Produced by Qwen3-235B-2507}
	\centering
	\begin{tikzpicture}[
		node distance=8mm,
		box/.style={draw, rounded corners, inner sep=6pt},
		service/.style={box, minimum height=12mm, text width=3.5cm, align=center},
		terminal/.style={box, draw=jupyter, text width=2.8cm, font=\ttfamily\small},
		ssh/.style={->, >=stealth, thick, orange, dashed},
		slurmjob/.style={->, >=stealth, thick, slurm, line width=1.5pt},
		browser/.style={circle, draw=jupyter, minimum size=8mm}
		]
		
		% ======== NODES ========
		% cnode07 (left)
		\node (cnode07) [service, fill=blue!5] {
			\textbf{cnode07} \\ 
			\footnotesize{3x A30 GPUs}
		};
		
		% Docker container inside cnode07
		\node (docker) [service, fill=docker!10, below=of cnode07, yshift=5mm] {
			\textbf{Docker Container} \\
			\footnotesize{JupyterHub Spawner}
		};
		
		% JupyterLab UI
		\node (jlab) [terminal, fill=white, above right=5mm and -5mm of docker] {
			\small\textcolor{jupyter}{\textbf{JupyterLab}} \\
			\footnotesize{(in browser)}
		};
		\node [browser] at (jlab.north west) [xshift=2mm, yshift=-2mm] {};
		
		% Terminals in JupyterLab
		\node (term1) [terminal, below=of jlab, yshift=-3mm] {
			\footnotesize{Terminal 1:} \\
			\texttt{\$ ssh cnode07}
		};
		\node (term2) [terminal, right=5mm of term1] {
			\footnotesize{Terminal 2:} \\
			\texttt{\$ qwen-cli ...}
		};
		
		% cnode08 (right)
		\node (cnode08) [service, fill=blue!5, right=4.5cm of cnode07] {
			\textbf{cnode08} \\ 
			\footnotesize{4x A100 GPUs}
		};
		
		% Llama.cpp server
		\node (llama) [terminal, fill=white, above=5mm of cnode08] {
			\small\textcolor{slurm}{\textbf{llama.cpp server}} \\
			\footnotesize{Qwen-Coder:12434}
		};
		
		% ======== CONNECTIONS ========
		% Browser to JupyterLab
		\draw [->, >=stealth, thick, jupyter] 
		(jlab.south) ++(0,-2mm) -- ++(0,-5mm) node[midway, right] {\scriptsize 1. User login};
		
		% Terminal 1: SSH to host
		\draw [ssh] (term1) -- ++(0,-8mm) -| 
		node[pos=0.7, below] {\scriptsize 2. Escape container} 
		(cnode07.south);
		
		% Slurm job launch
		\draw [slurmjob] (cnode07.east) -- ++(15mm,0) |- 
		node[pos=0.3, above] {\scriptsize 3. \texttt{srun llama-server}} 
		(llama.west);
		
		% CLI connection to server
		\draw [slurmjob] (term2.south) ++(0,-2mm) -- ++(0,-5mm) -| 
		node[pos=0.7, below] {\scriptsize 4. Connect to remote} 
		(llama.south);
		
		% ======== LABELS ========
		\node [anchor=north, font=\small\bfseries] at (cnode07.south) {Host OS};
		\node [anchor=north, font=\small\bfseries] at (docker.north) {JupyterHub};
		\node [anchor=south, font=\small, text=slurm] at (llama.north) {Slurm Job};
		
		% ======== LEGEND ========
		\begin{scope}[shift={(6,-2.5)}, font=\scriptsize]
			\node (sshicon) [circle, draw=orange, fill=orange!20, inner sep=1pt] {\tiny SSH};
			\node [right=2mm of sshicon] {Container escape};
			
			\node (slurmicon) [below=2mm of sshicon, circle, draw=slurm, fill=slurm!20, inner sep=1pt] {\tiny S};
			\node [right=2mm of slurmicon] {Slurm job};
		\end{scope}
		
	\end{tikzpicture}
	
	\vspace{3mm}
	\footnotesize
	\textbf{Sequence:} 
	1. User accesses JupyterLab via browser 
	$\rightarrow$ 
	2. SSH from container to host OS 
	$\rightarrow$ 
	3. Launch server on cnode08 via Slurm 
	$\rightarrow$ 
	4. CLI connects to remote server
\end{frame}


\begin{frame}{Control Flow: Distributed LLM Inference via Slurm}
	\centering
	\begin{tikzpicture}[
		node distance=8mm,
		box/.style={draw, rounded corners, inner sep=6pt},
		service/.style={box, minimum height=12mm, text width=3.5cm, align=center},
		terminal/.style={box, draw=jupyter, text width=2.8cm, font=\ttfamily\small},
		ssh/.style={->, >=stealth, thick, orange, dashed},
		slurmjob/.style={->, >=stealth, thick, slurm, line width=1.5pt},
		browser/.style={circle, draw=jupyter, minimum size=8mm}
		]
		
		% ======== NODES ========
		% cnode07 (left)
		\node (cnode07) [service, fill=blue!55] {
			\textbf{cnode07} \\ 
			\footnotesize{3x A30 GPUs}
		};
		
		% Docker container inside cnode07
		\node (docker) [service, fill=docker!64, below left=of cnode07, yshift=-1mm] {
			\textbf{Docker Container} 			
		};
		
		% JupyterLab UI
		\node (jlab) [terminal, fill=white, above=10mm of docker] {
			\small\textcolor{jupyter}{\textbf{JupyterLab}} \\
			\footnotesize{(in browser)}
		};
		
		\node (user) [browser] at (jlab.north) [xshift=0mm, yshift=16mm] {
			\textbf{user}
		};
		
		% Terminals in JupyterLab
		\node (term1) [terminal, right=of docker, yshift=-3mm] {
			\footnotesize{Terminal 1:} \\
			\texttt{\$ ssh cnode07}
		};
		\node (term2) [terminal, right=5mm of term1] {
			\footnotesize{Terminal 2:} \\
			\texttt{\$ qwen-cli ...}
		};
		
		% cnode08 (right)
		\node (cnode08) [service, fill=blue!55, right=of cnode07] {
			\textbf{cnode08} \\ 
			\footnotesize{4x A100 GPUs}
		};
		
		% Llama.cpp server
		\node (llama) [terminal, fill=white, above=5mm of cnode08] {
			\small\textcolor{slurm}{\textbf{llama.cpp server}} \\
			\footnotesize{Qwen-Coder:12434}
		};
		
		% ======== CONNECTIONS ========
		% Browser to JupyterLab
		\draw [->, >=stealth, thick, jupyter] 
		(user.south) ++(0,0mm) -- ++(0,-5mm) node[midway, right] {\scriptsize 1. User login};
		
		\draw [->, >=stealth, thick, jupyter] 
		(jlab.south) -- (docker.north) node[midway] {\scriptsize JupyterHub Spawner};
		
		
		% Terminal 1: SSH to host
		\draw [ssh] (term1) -- ++(0,-8mm) -| 
		node[pos=0.7, above] {\scriptsize 2. Escape container} 
		(cnode07.south);
		
		% Slurm job launch
		\draw [slurmjob] (cnode07.east) -- ++(15mm,0) |- 
		node[pos=0.3, above] {\scriptsize 3. \texttt{srun llama-server}} 
		(llama.west);
		
		% CLI connection to server
		\draw [slurmjob] (term2.south) ++(0,-2mm) -- ++(0,-5mm) -| 
		node[pos=0.7, below] {\scriptsize 4. Connect to remote} 
		(llama.south);
		
		% ======== LABELS ========
		\node [anchor=north, font=\small\bfseries] at (cnode07.south) {Host OS};
		\node [anchor=south, font=\small\bfseries] at (jlab.north) {JupyterHub};
		\node [anchor=south, font=\small, text=slurm] at (llama.north) {Slurm Job};
		
		% ======== LEGEND ========
		\begin{scope}[shift={(-2,2.5)}, font=\scriptsize]
			\node (sshicon) [circle, draw=orange, fill=orange!20, inner sep=1pt] {\tiny SSH};
			\node [right=2mm of sshicon] {Container escape};
			
			\node (slurmicon) [below=2mm of sshicon, circle, draw=slurm, fill=slurm!20, inner sep=1pt] {\tiny S};
			\node [right=2mm of slurmicon] {Slurm job};
		\end{scope}
		
	\end{tikzpicture}
	
	\vspace{3mm}
	\footnotesize
	\textbf{Sequence:} 
	1. User accesses JupyterLab via browser 
	$\rightarrow$ 
	2. SSH from container to host OS 
	$\rightarrow$ 
	3. Launch server on cnode08 via Slurm 
	$\rightarrow$ 
	4. CLI connects to remote server
\end{frame}


%%%%%%%	
	}

	
\end{document}




