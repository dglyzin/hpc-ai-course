\documentclass[aspectratio=169]{beamer}
\usepackage[utf8]{inputenc}
\usepackage[russian]{babel}
\usepackage{tikz}
\usepackage{graphicx}
\usepackage{fontspec}
\usepackage{adjustbox}
\usepackage[inkscapeversion=1.2]{svg} 
\usetikzlibrary{shadings}

\beamertemplatenavigationsymbolsempty

% Set the background color
% #000A1E
\definecolor{sitebackground1}{RGB}{0,10,30}
% #040714
\definecolor{sitebackground2}{RGB}{4,7,20}
% #66c6ea
\definecolor{logocolor}{RGB}{102,198,234}



\definecolor{darkblue}{RGB}{25,28,53}
\definecolor{baseblue}{RGB}{19,80,158}
\definecolor{additionalblue}{RGB}{47,127,193}

\definecolor{baselightblue}{RGB}{64,176,228}
\definecolor{additionallightblue}{RGB}{102,197,238}

\definecolor{baselightblue}{RGB}{64,176,228}

\setbeamercolor{background canvas}{bg=baseblue}

\setbeamercolor{normal text}{fg=white}
\setbeamercolor{title}{fg=white}
\setbeamercolor{frametitle}{fg=white}
\setbeamercolor{itemize item}{fg=logocolor}
\setbeamercolor{block title}{fg=logocolor}


% Load custom fonts
\newfontfamily\headerfont{Unageo}[
Path = ../fonts/,
Extension = .ttf,
UprightFont = *-Regular,
BoldFont = *-Bold,
ItalicFont = *-Regular-Italic,
]

\setsansfont{Montserrat}[
Path = ../fonts/,
Extension = .ttf,
UprightFont = *-Regular,
BoldFont = *-Bold,
ItalicFont = *-Italic,
]

% Set title and header font to Unageo
\setbeamerfont{title}{family=\headerfont, series=\bfseries}
\setbeamerfont{frametitle}{family=\headerfont, series=\bfseries}

% Command to easily adjust background images
\newcommand{\backgroundimages}[4]{%
	\begin{tikzpicture}[remember picture, overlay]
		% Left background image
		\node[anchor=south west, inner sep=0] at ([xshift=#1-0.3cm]current page.south west) 
		{\includesvg[width=0.3\paperwidth, height=0.3\paperheight]{#2}};
		
		% Right background image
		\node[anchor=north east, inner sep=0] at ([xshift=#1+0.3cm]current page.north east) 
		{\includesvg[width=0.3\paperwidth, height=0.3\paperheight]{#3}};
		
		% Center background image
		\node[anchor=south, inner sep=0] at ([xshift=#1-4cm]current page.south) 
		{\includesvg{#4}};
	\end{tikzpicture}%
}

\newcommand{\backgroundgradient}{%
	\begin{tikzpicture}[remember picture, overlay]		
		\node[anchor=south, inner sep=0] at ([xshift=-4cm]current page.south) 
		{\includesvg{Ellipse_4.svg}};
	\end{tikzpicture}%
}

\setbeamertemplate{background canvas}{
	\begin{tikzpicture}[remember picture, overlay]
		\shade[left color=baseblue, right color=darkblue] 
		(current page.south west) rectangle (current page.north east);
		
		\node[anchor=south west, inner sep=0] at ([xshift=-0.3cm]current page.south west) 
		{\includesvg[width=0.3\paperwidth, height=0.3\paperheight]{Group_4.svg}};
		
		% Right background image
		\node[anchor=north east, inner sep=0] at ([xshift=0.3cm]current page.north east) 
		{\includesvg[width=0.3\paperwidth, height=0.3\paperheight]{Group_5.svg}};
				
	\end{tikzpicture}
}

% Title page
\title{СУПЕРКОМПЬЮТЕРНЫЕ ТЕХНОЛОГИИ }
\subtitle{И ОСНОВЫ ИСКУССТВЕННОГО ИНТЕЛЛЕКТА}
\author{Дмитрий Глызин}
\institute{Акселкомп / ЯрГУ}
%\date{\today}
\date{Осень 2025}

\begin{document}
	
	% First slide with title
	{
		%\setbeamertemplate{background}{
		%	\backgroundimages{0cm}{Group_4.svg}{Group_5.svg}{Ellipse_4.svg}
		%}
		\begin{frame}
			\titlepage
		\end{frame}
	}
	
	% Section slide
	\section{Introduction}
	{
		%\setbeamertemplate{background}{
		%	\backgroundgradient
		%}
		\begin{frame}
			\frametitle{ОБ ЭТОМ КУРСЕ}
			\begin{block}{Преподаватели}
			\begin{itemize}
				\item Глызин Дмитрий Сергеевич --- лекции, семинары
				\item Куракин Павел Александрович --- практические занятия				
			\end{itemize}			
		    \end{block}
		    
		    \begin{block}{Занятия}
		    	\begin{itemize}
		    		\item Лекции, 1 пара в неделю. Математические основы машинного обучения и искуственного интеллекта. Статистическое машинное обучение. Теория Вапника-Червоненкиса.
		    		\item Семинары, 1 пара в две недели. Обсуждение статей в области ИИ, актуальных практических задач, продвинутых алгоритмов. 
		    		\item Практические занятия, 1 пара в неделю. Алгоритмы ML и DL, их реализации, анализ датасетов.
		    	\end{itemize}			
		    \end{block}
		    
		    
		\end{frame}
		
		\begin{frame}
			\frametitle{ОБ ЭТОМ КУРСЕ}			
			\begin{block}{Зачет}
				\begin{itemize}
					\item Результаты трех контрольных работы выше 75%					
					\item Выступление на семинаре 
					\item Проект по нейросетям
				\end{itemize}			
				Более сложные проекты могут снизить требования по остальным пунктам.
			\end{block}
		\end{frame}
		
		\begin{frame}
			\frametitle{ЗАЧЕМ?}			
				\begin{itemize}
					\item Зачем курс, если есть множество онлайн-курсов?
					\item Зачем курс, если нейросеть сама ответит на все вопросы и сгенерирует код?
				\end{itemize}			
		\end{frame}
		
	}
	
	% Content slide
	\section{supercomputing}
	{
		%\setbeamertemplate{background}{
		%	\backgroundimages{0cm}{Group_4.svg}{Group_5.svg}{Ellipse_4.svg}
		%}
		\begin{frame}
			\frametitle{ПРОГРЕСС}		
			\begin{block}{Прогресс в XX веке}
				\begin{itemize}
					\item 1952: "Кибернетика - “наука” мракобесов"
					\item 1960-1990: работы Вапника и Червоненкиса
					\item 1970: открытие факультета Вычислительной математики и кибернетки МГУ									
				\end{itemize}
					\end{block}
			\begin{block}{Прогресс на моей памяти}
				\begin{itemize}
					\item 1999: ночной интернет домой по телефонной линии вскладчину
					\item 2017: Национальный суперкомьютерный форум в Переславле под руководством С.М. Абрамова
					\item 2025: интеллектуальный ассистент в кармане		
					\item 2025: С.М. Абрамов осужден за финансирование экстремистской организации			
				\end{itemize}
			\end{block}	
				
		
		
		\end{frame}
	}
	\section{TOP-500}
	{
		%\setbeamertemplate{background}{
		%	\backgroundimages{0cm}{Group_4.svg}{Group_5.svg}{Ellipse_4.svg}
		%}
		
		
		\begin{frame}
			%\begin{tikzpicture}[remember picture, overlay]
				% First, draw the white rectangle at a specific location
			%	\fill[white] (current page.west) ++(1cm,-1cm) rectangle ++(5cm,-3cm);
			%\end{tikzpicture}
			
			\frametitle{РЕЙТИНГ СУПЕРКОМПЬЮТЕРОВ TOP-500}
						
			%\colorbox{white}{\includegraphics[page=1, trim=25 370 20 185, clip, width=\textwidth+1]{TOP500_202506_Poster.pdf}}
			\begin{tikzpicture}[remember picture,overlay]
			\node[anchor=west, inner sep=0] at ([xshift=0.cm]current page.west) 
			{\colorbox{white}
				{\includegraphics[page=1, trim=25 370 20 185, clip, width=\paperwidth]{TOP500_202506_Poster.pdf}}};
			\end{tikzpicture}
			
			%\vspace{1cm}
			%\centering
			%https://top500.org
		\end{frame}
        
       
		\begin{frame}
		
			\frametitle{СУПЕРКОМПЬЮТЕР}
				\begin{block}{Компоненты}
				\begin{itemize}
					\item узлы
					\item процессоры
					\item память
					\item ускорители 
					\item память ускорителя
					\item интерконнект					
				\end{itemize}
			\end{block}	
		\end{frame}

	}
	
	
	% Conclusion slide
	\section{Conclusion}
	{
		
		
		\begin{frame}
			\frametitle{TOP-500 ИЮНЬ 2025}
			\begin{itemize}
				\item 1 El Capitan (Ливерморская лаборатория) 1 742 000 TFlop/s
				\item 4 Jupiter (Исследовательский центр Юлиха) 793 400 TFlop/s
				\item 21 Sunway TaihuLight 93 000 TFlop/s (Национальный суперкомпьютерный центр в Уси) 93014 TFlop/s (без GPU)
				\item 75 Червоненкис (Яндекс)  21 530 TFlop/s
				\item 102 Галушкин (Яндекс) 
				\item 120 Ляпунов (Яндекс)
				\item 125 Кристофари Нео (Сбер)
				\item 201 Кристофари (Сбер)
				\item 495 Ломоносов 2 (МГУ)  2 478 TFlop/s
				
			\end{itemize}
			
		\end{frame}
		
		\begin{frame}
			\frametitle{ТЕРАФЛОПСЫ}
			\begin{itemize}
				\item flops = floating point operation per second									
				\item Стандартизованный бенчмарк для реальной производительности
				\item Тенденция к уменьшению точности и увеличению массивного параллелизма
			\end{itemize}	
			
		\end{frame}
		
		
		\begin{frame}
			\frametitle{ФОРМАТЫ ЧИСЕЛ}
			\begin{tikzpicture}[remember picture,overlay]
				\node[anchor=west, inner sep=0] at ([xshift=4.cm]current page.west) 
				{
					{\includegraphics[width=7cm]{float.png}}};
			\end{tikzpicture}
			
		\end{frame}
		
		\begin{frame}
			\frametitle{ОБУЧЕНИЕ LLM}
			\begin{itemize}
				\item Требует огромных ресурсов				
				\item Амодеи: на оборудовании, потребовавшемся для обучения llm, можно запустить миллион инференсов этой llm				
				\item Маск: не хватает трансформаторов для обучения трансформеров
			\end{itemize}	
			
		\end{frame}
		
		\begin{frame}
			
		\frametitle{ОБУЧЕНИЕ LLM}
		   Пример темы для выступления на семинаре: собрать информацию из открытых данных, фигурируют ли в TOP-500 машины, на которых обучались известные LLM
		   
	    \end{frame}	
		
		\begin{frame}
			\frametitle{СУПЕРДАННЫЕ}
			\begin{itemize}
				\item anna’s archive: 1600TB включая нераспознанные
				\item stackoverflow / stackexchange: несколько TB текста
				\item arxiv.org: 6TB
				\item Wikipedia: 200TB с медиафайлами
				\item reddit: более 20TB с картинками
				
			\end{itemize}		
		
		\end{frame}
		
		
	}
	
\end{document} 
